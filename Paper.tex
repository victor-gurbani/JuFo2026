\documentclass[12pt,a4paper]{article}
\usepackage{amsmath,amssymb,amsthm}
\usepackage{graphicx}
\usepackage[margin=1in]{geometry}
\usepackage{enumitem}
\usepackage{hyperref}
\usepackage{url}
\usepackage{tikz}
\usepackage{tikz-cd}
\usepackage{tikz-3dplot}
\usetikzlibrary{angles,quotes,arrows.meta,calc,babel,3d,positioning}
\usepackage{lipsum}
\usepackage{listings}
\usepackage{fancyhdr}
\usepackage{xcolor} % Added for colors
\usepackage{titlesec} % Added for title formatting

% Define colors
\definecolor{titlecolor}{RGB}{0, 51, 102}

% Consistent theorem/proof environments
\newtheorem{theorem}{Theorem}
\newtheorem{lemma}[theorem]{Lemma}
\newtheorem{corollary}[theorem]{Corollary}
\theoremstyle{definition}
\newtheorem{definition}[theorem]{Definition}
\newtheorem{problem}{Problem}

% Set up headers
\pagestyle{fancy}
\fancyhf{} % Clear all header and footer fields
\fancyhead[L]{Victor Gurbani}
\fancyhead[R]{JuFo 2026}
\fancyfoot[C]{\thepage} % Page number in center of footer

% Define a separate first page style with no numbers
\fancypagestyle{firstpage}{
    \fancyhf{} % Clear all header and footer fields
    \renewcommand{\headrulewidth}{0pt} % Remove header rule
    \renewcommand{\footrulewidth}{0pt} % Remove footer rule
}

% Create a custom title command
\renewcommand{\maketitle}{
    \begin{titlepage}
        \centering
        \vspace*{1cm}
        {\color{titlecolor}\rule{\linewidth}{1pt}}
        \vspace{1.5cm}
        
        {\fontsize{28}{34}\selectfont\color{titlecolor}\textbf{JuFo 2026}\par}
        
        \vspace{1.5cm}
        {\color{titlecolor}\rule{\linewidth}{1pt}}
        \vspace{2cm}
        
        {\Large\textbf{Victor Gurbani}\par}
        \vspace{0.5cm}
        {\large\today\par}
        
        \vfill
        
        % Optional: Add a simple decorative element
        \begin{tikzpicture}[remember picture, overlay]
            \draw[color=titlecolor, line width=0.5pt] 
                ($(current page.center) + (-3,0)$) -- ($(current page.center) + (3,0)$);
        \end{tikzpicture}
    \end{titlepage}
}

\begin{document}

\maketitle
\setcounter{page}{1}
\pagenumbering{arabic}
\newpage

\section{Abstract}
I curated a license-safe, composer-balanced corpus of 124 solo piano works spanning Bach, Mozart, Chopin, and Debussy from the 254{,}077-score PDMX archive, then engineered a three-pillar analytic pipeline that quantifies harmonic, melodic, and rhythmic practice per composer. Feature extraction with \texttt{music21} yields 36 descriptors whose distributions expose stylistic continuities and divergences, while automated significance tests highlight which metrics reliably separate eras. Interactive embeddings and colour-coded MusicXML renderings connect corpus-scale trends to score-level evidence: principal-component analysis shows Chopin bridging classical clarity and impressionist colour, whereas annotated scores reveal how our classifiers surface dissonances and chromatic harmonies directly inside MuseScore. The resulting dataset, figures, and tooling provide an integrated foundation for comparative analysis and reproducible storytelling about stylistic evolution from the Baroque through early modernism.

\section{Two-Page Summary of Findings}

\subsection{Balanced Corpus and Parsing Pipeline}
Starting with 254{,}077 PDMX entries, I filtered by licensing, originality, instrumentation metadata, and a composer-normalisation scheme to isolate a strictly solo-piano subset. The final cohort contains 30 works per target composer (120 total), covering 18{,}925 measures and approximately 63.6k quarter notes (\textasciitilde11.6 listening hours at 90~BPM). JSON metadata caching, arrangement detection, and instrumentation cross-checks ensure the remaining scores are genuine solo piano pieces rather than reductions or ensemble arrangements. A companion parsing script wraps disparate \texttt{music21} outputs (Score, Part, Opus) into a consistent \texttt{stream.Score} object, extracting measures, parts, and durations to produce reusable summaries under \texttt{data/parsed/}. This structural baseline prevents redundant MusicXML reads during later stages and enables quick sanity checks of corpus balance.

\subsection{Feature Engineering Across Three Pillars}
I engineered 36 descriptors: 16 harmonic, 11 melodic, and 9 rhythmic metrics. Harmonic analysis chordifies each score, computes chord-quality percentages, harmonic density, dissonance classifications (passing tones, appoggiaturas, other), and Roman numeral trends. Melodic features cover ambitus, interval statistics, leap ratios, pitch-class entropy, and soprano/bass interaction metrics (contrary, parallel, oblique motion). Rhythmic descriptors quantify duration spread, syncopation, downbeat emphasis, sliding-window entropy, micro-density of fast notes, and cross-hand subdivision mismatches. Each extractor offers CLI flags for limits, cached reuse, and optional boxplot generation, creating CSV outputs under \texttt{data/features/} and supporting documentation such as \texttt{Harmonic\_Features.md} and \texttt{Melodic\_Features.md}.

\subsection{Statistical Separation of Styles}
Running one-way ANOVA followed by Tukey HSD comparisons across the combined feature set surfaced 27 significant metrics spanning 56 composer pairings. This however had an $84\%$ of having a true positive. After a Bonferroni correction it decreased to 11 features, while with a looser FDR correction, it stayed at 26 significant metrics. Registral span (\emph{pitch\_range\_semitones}) and dissonance usage clearly separate Romantic and Impressionist writing from earlier styles (F=39.66, p\textless{}$7\times10^{-18}$). Rhythmic metrics such as \emph{std\_note\_duration} and \emph{rhythmic\_pattern\_entropy} differentiate Chopin and Debussy from Bach and Mozart, while syncopation ratios remain a defining trait for Debussy. Figure~\ref{fig:anova} summarises the strongest omnibus tests, while Figure~\ref{fig:pair-heatmap} maps how often each composer pair diverges.

\begin{figure}[h]
    \centering
    \includegraphics[width=0.85\textwidth]{figures/significance/top_anova_bar.png}
    \caption{Top ANOVA hits across harmonic, melodic, and rhythmic feature families.}
    \label{fig:anova}
\end{figure}

\begin{figure}[h]
    \centering
    \includegraphics[width=0.85\textwidth]{figures/significance/tukey_pair_heatmap.png}
    \caption{Count of statistically significant features per composer pairing (Tukey HSD).}
    \label{fig:pair-heatmap}
\end{figure}


\newpage

\subsection{Interpreting the Embedding Landscape}
Principal-component embeddings clarify how stylistic traits blend. After standardising the feature matrix and excluding raw count columns, PCA explains 47.3\% of the variance in the first three components. Composer centroids at (Mozart~\textminus2.05, 0.82, \textminus0.45), (Bach~\textminus0.27, \textminus1.06, \textminus0.91), (Chopin~0.41, \textminus0.02, 0.47), and (Debussy~1.92, 0.26, 0.90) reflect the axes' musical meaning: PC1 tracks chromatic density, leap ratio, and dissonance; PC2 contrasts long, downbeat-emphasised phrases with dense note streams; PC3 rewards oblique motion, cross-rhythms, and registral spread. Chopin bridges classical clarity and impressionist colour by sharing cadential discipline with Bach/Mozart (PC2) while adopting chromatic and rhythmic innovations that align him with Debussy (PC1/PC3). Debussy extends these traits, occupying a distinct ``third slot'' beyond Chopin. Interactive HTML views provide both point clouds and Gaussian ``composer clouds'' for browser-based exploration, augmented with directional lighting to enhance depth perception.

\begin{figure}[htbp]
    \centering 
    \includegraphics[width=0.85\textwidth]{figures/embeddings/composer_clouds_3d.png}
    \caption{3D PCA embedding showing composer clouds and stylistic separation across the first three principal components.}
    \label{fig:composer-clouds}
\end{figure}

\newpage

\subsection{Score-Level Evidence via Annotation}
To connect statistical findings to notation, the annotation pipeline enriches MusicXML files with coloured noteheads, lyric labels, chord symbols, and text expressions. Passing tones (orange), appoggiaturas (violet), other dissonances (red), and chromatic harmonies (turquoise) are flagged directly in the score. Chord symbols mirror Roman numeral analyses, and fallbacks ensure MuseScore displays readable figures even when automatic chord spelling fails. Batch scripts regenerate flagship works per composer and optionally invoke MuseScore's CLI to export PDF/PNG renditions, facilitating peer review without specialised software.

\subsection{Implications and Reuse}
Taken together, the curated dataset, multi-faceted feature suite, statistical evidence, and visual tooling deliver a reproducible laboratory for studying stylistic evolution from Baroque counterpoint to Impressionist colour. Analysts can reuse the CLI suite to test new hypotheses, integrate additional composers, or layer machine-learning models atop the harmonically balanced feature space. The documentation in and supporting markdown files captures both methodology and interpretive narratives, making the project portable for future musicology research.



\end{document}